The PLATO--Mission from the European Space Agency (ESA) is the successor of the Kepler and Corrol missons and aims to find more planets outside our solar system. Therefore the satellite will carry several cameras, which are pointed on as many stars as possible. Over a period of time, the luminosity of each star will be measured and saved. This data can be used to create light curves of single stars. Periodic peaks in those curves indicate that there are planets around the corrisponding stars. This is called the transit method and will be explained later in greater detail. 
\newline 
The success of the mission is highly dependant on a stable position of the satellite in space. Disturbances have to be countered with thrusters. For the exact control of those thrusters, a fine guidance system (FGS) is developed by the DLR Berlin. This is an optical system which will depend on the optical input of the onboard cameras of the satellite. It will ensure that the satellite will be in place over a long period of time with neglectable deviation. 
\newline 
It's not easy to test such applications, since it is not feasable to build and launch a prototype of the mission. On the other hand, it is of utmost importance to ensure before the start of the mission, that all systems and especially the fine guidance system work as they should. Therefore simulators are needed, which generate data approximated with all known sources of disturbance. Simulators allow to study the scientific community to study the performance of the instrument, its noise source response and the data quality. For the FGS there are two simulators used at the time the parallel developed DLR simulator, tailored for this exact task and the more general and more independantly developed PLATOSim simulator by the the KU Leuven.
\newline
Goal of this report is the summary of all steps taken to test the fgs and make sure all requirements for the mission ahead are met. Within this document the planned methods for detecting exoplanets and the according hardware of the PLATO--Mission are described. Next, the effects and probable sources of disturbances, as well as their possible influence is discussed. The third chapter deals with the used simulators. Their architecture and their methods of tackling the problem are described. The fourth chapter is about the generation of output data using the simulators and their use for the fgs. Strengths and weaknesses are discussed. The last Chapter is a summary of the work done and a prospect on the future testing of the fine guidance system.